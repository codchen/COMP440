\documentclass[11pt]{article}
\usepackage[left=2cm,top=2cm,right=2cm,bottom=2cm]{geometry}
\usepackage{graphicx}
\usepackage{setspace}
\usepackage{caption}
\usepackage{subcaption}
\usepackage{amsmath}
\usepackage{amsthm}
\usepackage[]{algorithm2e}
\usepackage{amsmath}
\usepackage[noend]{algpseudocode}
\usepackage{float}
\title{COMP 440 Homework 3}
\author{Tony Chen(xc12) and Adam Wang(sw33)}
\date{October 2016}
\begin{document}
\begin{onehalfspace}
\maketitle{}
\section{Sudoku and constraint satisfaction}
\begin{itemize}
	\item
	Constraint:
	\begin{enumerate}
		\item
		$X_{ij} = x_{ij}\forall \{$pre-filled cell $X_{ij}\}$, where $x_{ij}$ is $X_{ij}$'s pre-filled value
		\item
		$AllDiff(X_i)\forall\{$row $X_i\}$
		\item
		$AllDiff(X_j)\forall\{$column $X_j\}$
		\item
		$AllDiff(X_k)\forall\{$box $X_k\}$
	\end{enumerate}
	\item
	Forward checking eliminates all the inconsistent value w.r.t. the pre-filled cells from the domains of empty cells.\\
	For cell $X_{74}$, $3$ will be eliminated from its domain because of $X_{42}$ or $X_{95}$; $8$ will be eliminated from its domain because of $X_{44}$ or $X_{86}$ or $X_{78}$; $9$ will be eliminated from its domain because of $X_{54}$ or $X_{76}$; $7$ will be eliminated from its domain because of $X_{84}$ or $X_{77}$; $2$ will be eliminated from its domain because of $X_{85}$; $6$ will be eliminated from its domain because of $X_{94}$. So the domain of cell $X_{74}$ becomes $\{1,4,5\}$.
	\item
	Most-constrained variable heuristic will choose the variable with the smallest domain and thus enable the algorithm to detect failure sooner. This is particularly effective in the case of Sudoku because the constraint for each cell involves 20 other cells and the initial domain size is just 9; also since there are only 81 cells, the time and space complexity to get the cell with the smallest domain can be viewed as constant.\\
	For the example grid, the cell with the smallest domain are cell $X_{18}$ with domain $\{9\}$, cell $X_{64}$ with domain $\{5\}$, cell $X_{96}$ with domain $\{1\}$, and cell $X_{58}$ with domain $\{5\}$. So any one of them could be chosen to be assigned first.
	\item
	Because all other values in the same box as $X_{48}$ cannot have value $7$ and there must be a $7$ in that box.\\
	Yes it can.\\
	After forward checking, cell $X_{18}$ will have domain $\{9\}$, cell $X_{48}$ will have domain $\{4,5,7\}$, cell $X_{58}$ will have domain $\{5\}$, and cell $X_{98}$ will have domain $\{4,9\}$. Then we enforce arc consistency on $(X_{48},X_{58})$, which shrinks $X_{48}$'s domain to $\{4,7\}$; then we enforce arc consistency on $(X_{98},X_{18})$, which shrinks $X_{98}$'s domain to $\{4\}$; then we enforce arc consistency on $(X_{48},X_{98})$, which shrinks $X_{48}$'s domain to $\{7\}$. Therefore, after forward checking and enforcing arc consistency, the only value can be assigned to cell $X_{48}$ is $7$.
\end{itemize}
\section{Constraint satisfaction with non-binary factors}
\begin{itemize}
	\item
	Define $A$ as $(A_1,...,A_k)$ with domain $S$. $A$ has unary factor on itself such that $A_1,...,A_k$ are constrained by the $k$-nary factor over $X_1,...,X_k$. $A$ also has $k$ binary factors $A_i = X_i \forall 1 \leq i \leq k$. Note that the unary factor is not $k$-nary because it involve one variable with $k$ components of invariant values, instead of $k$ variable.\\
	\item
	 The binary factors are $A_1 = X_1$, $A_2 = X_2$, $A_3 = X_3$.\\
	 The initial domain of $A$: $\{(red,red,red),(red,red,blue),(red,red,green),(red,blue,red),$\\$(red,blue,blue),(red,blue,green),(red,green,red),(red,green,blue),(red,green,green),$\\$(blue,red,red),(blue,red,blue),(blue,red,green),(blue,blue,red),(blue,blue,blue),(blue,blue,green),$\\$(blue,green,red),(blue,green,blue),(blue,green,green),(green,red,red),(green,red,blue),$\\$(green,red,green),(green,blue,red),(green,blue,blue),(green,blue,green),(green,green,red),$\\$(green,green,blue),(green,green,green)\}$.
	 After enforcing consistent over the unary factor on each possible value from $A$'s original domain, $A$'s domain becomes $\{(red,green,green),(blue,green,green),$\\$(green,red,green),(green,blue,green),(green,green,red),(green,green,blue)\}$.
	 \item
	 
\end{itemize}
\end{onehalfspace}
\end{document}