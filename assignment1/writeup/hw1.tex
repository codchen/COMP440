\documentclass[11pt]{article}
\usepackage[left=2cm,top=2cm,right=2cm,bottom=2cm]{geometry}
\usepackage{graphicx}
\usepackage{setspace}
\usepackage{caption}
\usepackage{subcaption}
\usepackage{amsmath}
\usepackage{amsthm}
\usepackage[]{algorithm2e}
\usepackage{float}
\title{COMP 440 Homework 1}
\author{Tony Chen(xc12) and Adam Wang(sw33)}
\date{August 2016}
\begin{document}
\begin{onehalfspace}
    \maketitle
    \newpage{}
    \section{Modeling sentence segmentation as search}
    \begin{itemize}
        \item
        Suppose the sentence length is $L$, we define the state space model as following:
        \begin{itemize}
            \item
            $S = s_0\cup s_1\cup s_2\cup ... \cup s_L$, where, for $i = 0, 1, 2, ... L$, $s_i$ represents the state that exactly $i$ first characters have been segmented.
            \item
            $Action$ is defined as: for every word $w$ in $D$ that is a prefix from the $i$-th character in the sentence, $a_{i,w}$ is the action to jump to state $s_{i+len(w)}$.
            \item
            $Successor(s_i, a_{i,w}) = s_{i+len(w)}$
            \item
            $Cost(s_i, a_{i,w}) = 1$
            \item
            $s_{start} = s_0$
            \item
            $isGoal(s_i) = (s_i == s_L)$
        \end{itemize}
        \item
        *Since the paths of unweighted graph of states are directed and never going "backward", there's no cycle in the graph.
        \begin{itemize}
            \item
            Yes. BFS will be able to traverse the graph layer by layer from $s_0$ and once it reaches $s_L$, the current distance (number of layers traversed) is the minimum cost path. Time complexity will be $O(b^s)$ where $b$ is the average number of neighbours and $s$ is the number of layers from $s_0$ to $s_L$.
            \item
            No. DFS will not guarantee the distance we have when we find $s_L$ to be the minimum cost path.
            \item
            Yes. UFS will be able to find a least cost path from $s_0$ to $s_L$ in $O(L)$.
            \item
            Yes. A* with a consistent heuristic will be able to find a least cost path from $s_0$ to $s_L$ in $O(L)$.
            \item
            Yes. Bellman-Ford will be able to find a least cost path from $s_0$ to $s_L$ in $O(L^2|A|)$, where $|A|$ is the number of edges.
        \end{itemize}
        \item
            Modify $Cost$ function so that $Cost(s_i, a_{i,w}) = len(w) - 1$. \\
            BFS won't work now since it only works on unweighted graph. UFS, A*, and Bellman-Ford will still work since they all can generalize on weighted graph.
        \item
            Modify $S$ so that for a state $s_{i,last}$ where $i > 0$, exactly $i$ first characters have been segmented and the last segmented word is $last$. \\
            Modify $Cost$ function so that $Cost(s_0, a_{0,w}) = 0$ and $Cost(s_{i,last}, a_{i,w}) = fluency(last, w)$.
    \end{itemize}
    \newpage{}
    \section{Searchable Maps}
    \begin{itemize}
        \item
        We define the cost as the time required to travel from $s$ to $t$. Since we need a heuristic that never overestimates the cost, we define it as the lowest possible cost, which is obtained by a traveling along a straight line from $s$ to $t$ at the highest possible speed: \\
        \begin{eqnarray*}
            h(s, t) = \frac{G(s, t)}{S_H}
        \end{eqnarray*}
        \item
        The heuristic is defined as following:
        \begin{eqnarray*}
            h(s, t) = |T(s, L) - T(L, t)|
        \end{eqnarray*}
        \item
        Suppose the goal node is $t$, it is trivial that $h(t) = h_1(t) = h_2(t) = 0$.
        Consider any pair of node $n$ and $m$ where there is an action to get $m$ from $n$. Suppose $h_1(n) \geq h_2(n)$, then $h(n) = h_1(n)$ and there are two possibilities: first, $h_1(m) \geq h_2(m)$, which makes $h(m) = h_1(m)$ and obviously makes $h$ consistent on $n$ and $m$ ($h$ is exactly $h_1$); second, $h_1(m) < h_2(m)$, then since $h(n) = h_1(n) \leq Cost(n,m) + h_1(m) < Cost(n,m) + h_2(m)$, $h$ is also consistent on $n$ and $m$. So $h$ will always be consistent in this case\\
        Since $h_1$ and $h_2$ are symmetric, the above conclusion will also hold for $h_1(n) < h_2(n)$. So $h$ is always consistent.
        \item According to part c, the basic idea is to take the max of all heuristics:
        \begin{eqnarray*}
            h(s, t) = max(|T(s, L_1) - T(L_1, t)|, |T(s, L_2) - T(L_2, t)|, ... , |T(s, L_K) - T(L_K, t)|, \frac{G(s, t)}{S_H})
        \end{eqnarray*}
        \item
        For adding edges, $h$ will NOT remain consistent. Image the new edge draws a straight line from $s$ to $t$, then as long as the original path estimate of $h$ is not a straight line on its own, $h$ will overestimate, which makes it not only inconsistent but also not admissible. \\
        For removing edges, $h$ will still remain consistent. Since for any edge remains in the graph, none of the three parts in the inequality $h(m) \leq Cost(m, n) + h(n)$ will change. So $h$ will remain consistent.
    \end{itemize}
    \newpage{}
    \section{Designing Search Algorithms: Protein Folding}
    \begin{itemize}
    \item
    $S$ is composed by $s_{i,j,path}$'s which represents that the last "placed" residue has coordinates $(i, j)$ and that the path before this residue is recorded by $path$, which is a list of coordinates. Note that this residue is the $(len(path) + 1)$-th one in the sequence.
    \end{itemize}
\end{onehalfspace}
\end{document}